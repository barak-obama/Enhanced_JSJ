\documentclass[11pt,a4paper]{article}


\input{preamble.tex}


\title{Enhanced JSJ}

\author{Enhanced JSJ}
\date{}


\input{commands.tex}

%-----------------------
% Begin document
%-----------------------
\begin{document} 
\maketitle
\begin{abstract}
    This is a file intended for supporting the reading of \cite{Fiv25}
\end{abstract}

Questions to ask:
\begin{enumerate}
    \item How much of it can be generalized to general HHGs? Especially MCG.
    \item Can it be defined for limit groups over special groups? over HHGs?
\end{enumerate}


$G$ is a subgroup of RAAG $A_\Gamma$


\begin{definition}
    \begin{enumerate}
        \item A subgroup of a RAAG $A_\Gamma$ is called \textbf{parabolic} if it is a subgroup of $A_\Delta$ for some $\Delta \lneq \Gamma$, i.e. generated by a proper subset of the standard generators (\q{does it mean parabolic action on the contact graph? Wait, the action on the contact graph is acylindrical}).
        \item If $G\leq A_\Gamma$ then $H\leq G$ is $G-parabolic$ if there exists parabolic subgroup $P$ such that $H=G\cap P$
        \begin{equation*}
            \calp \rb{ G } = \cb{ H \leq G \mid \mbox{$H$ is $G$-parabolic} }
        \end{equation*}
        
        \begin{equation*}
            \hat{H} = \bigcap_{H \subseteq K\in \calp\rb{ G }} K
        \end{equation*}
        \item A centralizer in $G$ is a subgroup of the form $H=Z_G\rb{ A }$ for some $A\subseteq G$
        \begin{equation*}
            \calz \rb{ G } = \rb{ H \leq G \mid \mbox{$H$ is a centralizer in $G$} }
        \end{equation*}
        \item A subgroup $H\leq G$ is $p$-seperable if its closed in the pro-$p$ topology of $G$. Aquivenattly if for all $g\in G\\H$ there is finite $p$-group $F$ and $f:G\to F$ such that $f\rb{ g }\neq1$.
        \item 
        \begin{equation*}

            
        \cals \rb{ G } = \rb{ H \leq G \mid \mbox{ $H$ virtually splits  } }
        \end{equation*}
    \end{enumerate}
\end{definition}



\printbibliography




\end{document}